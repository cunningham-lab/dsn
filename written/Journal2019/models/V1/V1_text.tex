Dynamical models with two populations (excitatory (E) and inhibitory (I) neurons) of visual processing have been used to reproduce a host of experimentally documented phenomena in V1.   When an inhibition stabilized network (ISN, the I population stabilizes an otherwise unstable E population), these models exhibit the paradoxical effect \cite{tsodyks1997paradoxical}, selective amplification \cite{murphy2009balanced}, surround suppression \cite{ozeki2009inhibitory}, and  sensory integrative properties \cite{rubin2015stabilized}.  Since almost all I neurons fall into one of three classes (parvalbumin (P)-, somatostatin (S)-, and vasointestinal peptide (V)-expressing neurons) \cite{markram2004interneurons, rudy2011three}, theoretical neuroscientists look to extend these dynamical models to four populations \cite{litwin2016inhibitory}.  A current challenge in theoretical neuroscience is understanding the distributed role of inhibition stabilization across these inhibitory subtypes.  

These four populations exhibit neuron-type specific connectivity (Fig. 1A) \cite{pfeffer2013inhibition}, in which some populations do not project to others.  Since S and V are the only populations that mutually inhibit each other, a popular conceptualization is that S and V have winner-take-all dynamics.  In fact, evidence in mice suggests that V silences S when presented with large stimuli, and S silences V for small stimuli \cite{dipoppa2018vision}.  Here, we use DSNs to understand the possible sources of inhibition stabilization in this V1 model, when either S or V is inactive, selecting the weight matrix parameters as the free parameters of the DSN.  The behavior of the DSN sampled models is constrained to produce two things: 1.) a mean-zero distribution of ISN coefficients $\gamma(W) = 1 - f^{'}(f^{-1}(r_E(W)))W_{EE}$ with some variance, and 2.) $\alpha$-population silencing $r_\alpha(W) = 0$, for $\alpha \in \{ S, V \}$.  When $\gamma < 0$ the network is ISN, and not ISN otherwise.  Constraining the DSN behavior to a zero-mean distribution of ISN coefficients gives us samples of both ISN and non-ISN networks, optimizied to have greatest variety of stabilization motifs.

( -- paragraph about V1 analyses --)

